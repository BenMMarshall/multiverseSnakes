
\documentclass[10pt,a4paper]{article}
\usepackage{f1000_styles}

%% Default: numerical citations
% \usepackage[numbers]{natbib}

%% Uncomment this lines for superscript citations instead
% \usepackage[super]{natbib}

%% Uncomment these lines for author-year citations instead
% \usepackage[round]{natbib}
% \let\cite\citep

%% lines required to use a CSL style for references
% definitions for citeproc citations
\NewDocumentCommand\citeproctext{}{}
\NewDocumentCommand\citeproc{mm}{%
  \begingroup\def\citeproctext{#2}\cite{#1}\endgroup}
\makeatletter
 % allow citations to break across lines
 \let\@cite@ofmt\@firstofone
 % avoid brackets around text for \cite:
 \def\@biblabel#1{}
 \def\@cite#1#2{{#1\if@tempswa , #2\fi}}
\makeatother
\newlength{\cslhangindent}
\setlength{\cslhangindent}{1.5em}
\newlength{\csllabelwidth}
\setlength{\csllabelwidth}{3em}
\newenvironment{CSLReferences}[2] % #1 hanging-indent, #2 entry-spacing
 {\begin{list}{}{%
  \setlength{\itemindent}{0pt}
  \setlength{\leftmargin}{0pt}
  \setlength{\parsep}{0pt}
  % turn on hanging indent if param 1 is 1
  \ifodd #1
   \setlength{\leftmargin}{\cslhangindent}
   \setlength{\itemindent}{-1\cslhangindent}
  \fi
  % set entry spacing
  \setlength{\itemsep}{#2\baselineskip}}}
 {\end{list}}
\usepackage{calc}
\newcommand{\CSLBlock}[1]{\hfill\break#1\hfill\break}
\newcommand{\CSLLeftMargin}[1]{\parbox[t]{\csllabelwidth}{\strut#1\strut}}
\newcommand{\CSLRightInline}[1]{\parbox[t]{\linewidth - \csllabelwidth}{\strut#1\strut}}
\newcommand{\CSLIndent}[1]{\hspace{\cslhangindent}#1}

%% lines to get the code chunks working

%% lines to enable bulletpoints in a new notation style
\providecommand{\tightlist}{%
  \setlength{\itemsep}{0pt}\setlength{\parskip}{0pt}}

\begin{document}
\pagestyle{fancy}

\title{Testing the conclusions of snake habitat selection studies with a multiverse of analyses}
\author[1]{Benjamin Michael Marshall*}
\author[1]{Alexander Bradley Duthie**}
\affil[1]{Biological and Environmental Sciences, Faculty of Natural Sciences, University of Stirling, Stirling, FK9 4LA, Scotland, UK}

\affil[*]{\href{mailto:benjaminmichaelmarshall@gmail.com}{\nolinkurl{benjaminmichaelmarshall@gmail.com}}}
\affil[**]{\href{mailto:alexander.duthie@stir.ac.uk}{\nolinkurl{alexander.duthie@stir.ac.uk}}}

\maketitle
\thispagestyle{fancy}

\begin{abstract}

PLACEHOLDER ABSTRACT Possibly the best way to determine a snake's needs is to follow their movements. Once we have learnt of the snake's movements we can infer habitat requirements, behaviour, and potential threats. Combined, the movement data and inferences can inform decisions on snake conservation and human snake conflict. However, extracting useful information from snake movement requires many steps, from sampling to analysis. Other studies have shown that a single dataset can result in many different answers in the hands of different researchers, so how can we be confident the results from snake movement are leading to the correct decisions in snake conservation? We used a multiverse approach to explore thousands of ways of extracting the habitat preference estimates from the movement of simulated snakes with a pre-defined preference. We found that despite different sampling approaches, and completely different analysis methods, the vast majority of results agree and correctly identify habitat preference. The agreement between different habitat preference estimates tended to be better with more data, and when using more modern analysis methods. Now we plan to apply this multiverse of analyses to re-examine several previous studies of snake habitat preference from Thailand. We will examine how the published results compare to the thousands of other ways a researcher could have examined the snake movement data. Would certain analysis choices have led to a different conclusion and therefore a different conservation recommendation? We hope the answers to these questions will inform how confident we can be in the findings from snake movement studies and direct us towards more robust studies in the future.

\end{abstract}

\section*{Keywords}

Movement ecology, simulation, step selection function, poisson, habitat preference, habitat selection, animal movement, multiverse, research choice, researcher degrees for freedom, snakes, King Cobra, Burmese Python, Banded Krait, Malayan Krait

\clearpage
\pagestyle{fancy}

\section{Introduction}\label{introduction}

A key component of science is the continual reassessment of past work and findings (\citeproc{ref-alberts_self-correction_2015}{Alberts et al., 2015}).
Whether that takes the form of direct replications aiming to discover exactly how reliable previous work is, or more integrative approaches testing the edges of previous findings' generalisability and retesting questions in different study systems (\citeproc{ref-nakagawa_replicating_2015}{Nakagawa \& Parker, 2015}; \citeproc{ref-peterson_self-correction_2021}{Peterson \& Panofsky, 2021}).

Reassessments and replications --regardless of their position on the direct-quasi continuum-- can aid the formal and organic self-correcting process of science.
Initial findings set the stage for subsequent work, building momentum that can accelerate progress, but this momentum can be difficult to redirect if the initial impetus was misdirected (\citeproc{ref-jennions_relationships_2002}{Jennions \& Møller, 2002}; \citeproc{ref-barto_dissemination_2012}{Barto \& Rillig, 2012}) .
Therefore, checking and confirming results early is important; we can see this principle recognised in the peer review system itself.

Checking previous findings through replication can become more difficult in systems with high task uncertainty.
High task uncertainty systems --those that manifest high levels of uncontrollable stochasticity-- may make direct diagnostic replications impractical or impossible, and render the evidence from quasi-replications weaker (\citeproc{ref-peterson_self-correction_2021}{Peterson \& Panofsky, 2021}).
Ecological systems can be considered as generating high task uncertainty, with many interconnected elements, and when studying wild systems many of those elements are uncontrollable.

With ecological systems, such complexity and the difficulties in controlling experiments makes direct replications costly, potentially explaining their rarity (\citeproc{ref-kelly_rate_2019}{Kelly, 2019}).
When studying wild animals with a level of direct intervention, repeating experiments might be unethical due to the well-being costs (\citeproc{ref-Weatherhead2004}{Weatherhead \& Blouin-Demers, 2004}; \citeproc{ref-robstad_impact_2021}{Robstad et al., 2021}; \citeproc{ref-tomotani_great_2021}{Tomotani et al., 2021}; \citeproc{ref-portugal_externally_2022}{Portugal \& White, 2022}; \citeproc{ref-altobelli_methods_2022}{Altobelli et al., 2022}).

When faced with limited options for direct replications, an alternative, albeit not a replacement, would be to re-examine existing datasets.
Pooling old and new datasets, and reanalysing them, may provide opportunities for broader generalisations.

In some cases older data may have been collected and recorded in ways that enables completely fresh analysis (\citeproc{ref-kays_movebank_2022}{Kays et al., 2022}).
As methodologies develop, conceptualisations change, and computational power increases, new avenues for examining the same data may materialise (e.g., \citeproc{ref-noonan_effects_2020}{Noonan et al., 2020}).
As these new methods are developed and applied, we may see the conclusions based upon those data change.
There are a growing number of examples demonstrating that the analysis approach can alter the results (\citeproc{ref-salis_how_2021}{Salis, Lena \& Lengagne, 2021}; \citeproc{ref-desbureaux_subjective_2021}{Desbureaux, 2021}), and that the researchers themselves can be a key source of variation in analysis and results (\citeproc{ref-silberzahn_many_2018}{Silberzahn et al., 2018}; \citeproc{ref-huntingtonklein_influence_2021}{Huntington‐Klein et al., 2021}; \citeproc{ref-gould_same_2023}{Gould et al., 2023}).
These examples elegantly show the possible extent of technical uncertainty present in some systems.

Not all disciplines have explored the sources of uncertainty in findings equally.
Prudence would push for examination of uncertainty in all its forms, in particular for fields that already tackle high levels of uncertainty originating from a wild study system.
Movement ecology could be argued to exemplify such a field.
Animals are complex, existing in complex wild ecosystems, with individuality and personality (\citeproc{ref-stuber_spatial_2022}{Stuber, Carlson \& Jesmer, 2022}).
Depending on the research question, controls in movement ecology can be difficult to achieve, and replications difficult to justify given the strict ethical limitations on interventionist study.
Movement ecology has also seized the opportunities presented by technological developments, enabling higher resolution tracking of animal movements (e.g., GPS tracking) and more sophisticated analysis that can integrate the high dimensional data (e.g., x-y coordinates, time, acceleration, individual, other covariates of interest, \citeproc{ref-joo_recent_2022}{Joo et al., 2022}).

Personality and the repeatability of behaviours presents a key component to the uncertainty or variation when attempting to generalise.
However, here we turn to the technical uncertainty, the uncertainty originating from the researcher and how they approach the data.
Previous many analyst projects highlight the potential for analyst-side variation (\citeproc{ref-silberzahn_many_2018}{Silberzahn et al., 2018}; \citeproc{ref-huntingtonklein_influence_2021}{Huntington‐Klein et al., 2021}; \citeproc{ref-gould_same_2023}{Gould et al., 2023}), and previous multiverse explorations of movement ecology methods highlight the variation potentially presented within a synthetic movement dataset {[}chapter 2 and 3 preprints can be cited here when published in June{]}.
Here we take the multiverse approach further by applying it to a number of real case studies with the aim of exploring whether different analysis approaches could have altered the final general conclusions.

We selected a quartet of separate but connected movement ecology studies that attempt to disentangle the habitat selection exhibited by snakes in north-eastern Thailand.
All four cases focus on snakes that come into conflict with humans to some extent, either because of the risks posed by their venom (King Cobra Marshall et al. (\citeproc{ref-Marshall2018}{2019}) \& Marshall et al. (\citeproc{ref-marshall_no_2020}{2020}), Malayan Krait Hodges et al. (\citeproc{ref-hodges_malayan_2022}{2022}), Banded Krait Knierim et al. (\citeproc{ref-knierim_spatial_2019}{2019})), or because of their appetite for domestic livestock (Burmese python Smith et al. (\citeproc{ref-smith_native_2021}{2021})).
In all cases the habitat selection results could be used to guide snake conservation efforts, as well as interventions into human behaviour to mitigate human-snake conflict.
With these general goals in mind, we aim to re-examine the movement datasets using a multiverse of habitat selection analysis pathways to reveal whether the same data could lead to different conclusions.

\section{Methods}\label{methods}

\subsection{Study Location}\label{study-location}

All four case studies occurred in north eastern Thailand, within Nakhon Ratchasima province.
Three case studies (King Cobra, Burmese Python, Banded Krait) were conducted within the Sakaerat Biosphere Reserve.
The reserve comprises of three zones of management: core, buffer, and transitional.
The core is largely primary forest; the buffer surrounds the core and is comprised of forest regeneration efforts, whereas the transitional zone allows more development resulting in a mix of agriculture, settlements, and plantation forest.
Bisecting the transitional zone, and running adjacent to the protected forest areas is a four-lane highway connecting the city of Nakhon Ratchasima to Bangkok.
The case study (Malayan Krait) not in the Sakaerat Biosphere Reserve was undertaken nearer to Nakhon Ratchasima proper, on the Suranaree University of Technology campus.
The university campus is a mix of scrub forest, open lawn, university buildings, and homes.
Further details on the study sites' characteristics can be found in the original publications (\citeproc{ref-Marshall2018}{Marshall et al., 2019}, \citeproc{ref-marshall_no_2020}{2020}; \citeproc{ref-knierim_spatial_2019}{Knierim et al., 2019}; \citeproc{ref-smith_native_2021}{Smith et al., 2021}; \citeproc{ref-hodges_malayan_2022}{Hodges et al., 2022}).

\subsubsection{Study Species and Hypotheses}\label{study-species-and-hypotheses}

Snakes can be difficult to detect in wild scenarios (\citeproc{ref-Durso2015}{Durso \& Seigel, 2015}; \citeproc{ref-boback_use_2020}{Boback et al., 2020}), forcing a wider and more opportunistic suite of methods to gather adequate sample sizes.
In all the chosen case studies snakes were obtained for study using trapping arrays, active surveying, and notifications from locals.
The local notifications often arose from snakes entering human settlements, and a desire for the snake to be removed.

The four case studies cover four snake species, each with their own ecology and movements.

\paragraph{King Cobra}\label{king-cobra}

Marshall et al. (\citeproc{ref-Marshall2018}{2019}) and Marshall et al. (\citeproc{ref-marshall_no_2020}{2020}) are concerned with King Cobras (\emph{Ophiophagus hannah}).
King Cobras are a large (tracked individuals between 1.40 and 3.71m snout to vent length), diurnal, active foraging snake species that depredate snakes and monitor lizards (\citeproc{ref-Jones_supposed_2020}{Jones et al., 2020}).
While considered a predominately forest dwelling species (\citeproc{ref-Stuart2012}{Stuart et al., 2012}), they are known to make use of more human altered areas (\citeproc{ref-Whitaker2004}{Whitaker \& Captain, 2004}; \citeproc{ref-Rao2013}{Rao et al., 2013}; \citeproc{ref-jones_how_2022}{Jones et al., 2022}), which can lead to frequent human-snake conflict (\citeproc{ref-Shankar2013a}{Shankar et al., 2013}; \citeproc{ref-Marshall2018b}{Marshall et al., 2018}).
The extremely low occurrence of King Cobra bites in Thailand means that instances of human-snake conflict are primarily a conservation concern as opposed to human health (\citeproc{ref-Viravan1992}{Viravan et al., 1992}; \citeproc{ref-Pochanugool1998}{Pochanugool et al., 1998}).

Marshall et al. (\citeproc{ref-Marshall2018}{2019}) does not conclude on an actual selection, instead highlighting the King Cobras' excursions out of the protected forest.
Marshall et al. (\citeproc{ref-marshall_no_2020}{2020}) looks more specifically at selection, highlighting the importance of semi-natural areas that occupy the banks of irrigation canals and intersect the agricultural matrix surrounding the protected forest.
Therefore, we will pool both datasets and examine two non-mutually exclusive hypotheses that can be examined through a unified model.

H\textsubscript{OPHA1}: King Cobras select for semi-natural habitat.

H\textsubscript{OPHA2}: King Cobras select for forest habitat.

\paragraph{Burmese Python}\label{burmese-python}

Smith et al. (\citeproc{ref-smith_native_2021}{2021}) describe Burmese Python (\emph{Python bivittatus}) habitat selection and movement.
Burmese Pythons are large (tracked individuals between 2.21 and 3.09m snout to vent length), ambush predators capable of tacking prey over 100\% their own body mass (\citeproc{ref-bartoszek_natural_2018}{Bartoszek et al., 2018}) and impacting mammal populations (\citeproc{ref-dorcas_severe_2012}{Dorcas et al., 2012}).
The flexibility in regards to prey size means snakes of this size are inevitably drawn into conflict with humans over livestock, a pattern mirrored across the globe for large snakes (\citeproc{ref-Miranda2016}{Miranda, Ribeiro- \& Strüssmann, 2016}).

The conclusions of Smith et al. (\citeproc{ref-smith_native_2021}{2021}) on python habitat selection are not dissimilar to those made on King Cobras, with an active selection for areas near water.
The land classification used in Smith et al. (\citeproc{ref-smith_native_2021}{2021}) was slightly different to Marshall et al. (\citeproc{ref-marshall_no_2020}{2020}), grouping semi-natural areas with larger water bodies (e.g., agricultural ponds).

H\textsubscript{PYBI1}: Burmese Pythons select for areas near water.

\paragraph{Malayan Krait}\label{malayan-krait}

Hodges et al. (\citeproc{ref-hodges_malayan_2022}{2022}) examine a smaller species, the Malayan Krait (\emph{Bungarus candidus}).
The Malayan Kraits tracked were between 0.65 and 1.46m snout to vent, and all lived on a university campus.
Malayan Kraits, like many elapids, have a potent and medically significant venom; bites of Malayan Kraits can be fatal (\citeproc{ref-looareesuwan_factors_1988}{Looareesuwan, Viravan \& Warrell, 1988}; \citeproc{ref-searo_regional_office_for_the_south_east_asia_rgo_guidelines_2016}{South East Asia (RGO) \& Asia, 2016}).
They are (mostly) nocturnal and actively foraging (\citeproc{ref-hodges_deadly_2021}{Hodges et al., 2021}), known to depredate a wide range of prey (\citeproc{ref-kuch_notes_2001}{Kuch, 2001}; \citeproc{ref-hodges_diurnal_2020}{Hodges, D'souza \& Jintapirom, 2020}).

Unlike the other case studies, Hodges et al. (\citeproc{ref-hodges_malayan_2022}{2022}) is undertaken in a more urban environment.
The scale of the Malayan Krait movements meant the study was conducted at a finer spatial scale; habitat types are therefore more finely separated (e.g., buildings vs settlements).
The overall conclusions highlight two habitat types that are potentially being selected for, in contrast to the lack of selection for open areas.

H\textsubscript{BUCA1}: Malayan Kraits select for buildings and natural areas.

\paragraph{Banded Krait}\label{banded-krait}

Knierim et al. (\citeproc{ref-knierim_spatial_2019}{2019}) looked at a larger krait species, the Banded Krait (\emph{Bungarus fasciatus}).
Like its smaller cousin, the Banded Krait is also a nocturnal active forager, with a potent venom.
The Banded Krait is heavier-bodied and grows to longer lengths, tracked individuals ranging from 1.13 and 1.58 m snout to vent length.
However, unlike the Malayan Krait, the Banded Krait appears less tolerant of human disturbance in this region of Thailand and tends to have a more ophiophagus diet (\citeproc{ref-Knierim2017a}{Knierim, Barnes \& Hodges, 2017}).

Banded Kraits were entirely located in agricultural land, and like the other krait had movements more conducive to finer habitat classifications.
For example, field margins were found as a key nesting site (\citeproc{ref-knierim_spatial_2019}{Knierim et al., 2019}).
Knierim et al. (\citeproc{ref-knierim_spatial_2019}{2019}) shows that importance is reflected in the movements and habitat selection, as Banded Kraits follow the linear water or field margin features as opposed to the wider more exposed field areas.

H\textsubscript{BUFA1}: Banded Kraits select for waterways and field edges.

\section*{References}\label{references}
\addcontentsline{toc}{section}{References}

\phantomsection\label{refs}
\begin{CSLReferences}{1}{0}
\bibitem[\citeproctext]{ref-alberts_self-correction_2015}
Alberts B, Cicerone RJ, Fienberg SE, Kamb A, McNutt M, Nerem RM, Schekman R, Shiffrin R, Stodden V, Suresh S, Zuber MT, Pope BK, Jamieson KH. 2015. Self-correction in science at work. \emph{Science} 348:1420--1422. DOI: \href{https://doi.org/10.1126/science.aab3847}{10.1126/science.aab3847}.

\bibitem[\citeproctext]{ref-altobelli_methods_2022}
Altobelli JT, Dickinson KJM, Godfrey SS, Bishop PJ. 2022. Methods in amphibian biotelemetry: {Two} decades in review. \emph{Austral Ecology} 47:1382--1395. DOI: \href{https://doi.org/10.1111/aec.13227}{10.1111/aec.13227}.

\bibitem[\citeproctext]{ref-barto_dissemination_2012}
Barto EK, Rillig MC. 2012. Dissemination biases in ecology: Effect sizes matter more than quality. \emph{Oikos} 121:228--235. DOI: \href{https://doi.org/10.1111/j.1600-0706.2011.19401.x}{10.1111/j.1600-0706.2011.19401.x}.

\bibitem[\citeproctext]{ref-bartoszek_natural_2018}
Bartoszek I, Andreadis PT, Prokop-Ervin C, Patel M, Reed RN. 2018. Natural {History} {Note}: {Python} bivittatus ({Burmese} {Python}). {Diet} and {Prey} {Size}. \emph{Herpetological Review} 49:139--140.

\bibitem[\citeproctext]{ref-boback_use_2020}
Boback SM, Nafus MG, Yackel Adams AA, Reed RN. 2020. Use of visual surveys and radiotelemetry reveals sources of detection bias for a cryptic snake at low densities. \emph{Ecosphere} 11. DOI: \href{https://doi.org/10.1002/ecs2.3000}{10.1002/ecs2.3000}.

\bibitem[\citeproctext]{ref-desbureaux_subjective_2021}
Desbureaux S. 2021. Subjective modeling choices and the robustness of impact evaluations in conservation science. \emph{Conservation Biology} 35:1615--1626. DOI: \href{https://doi.org/10.1111/cobi.13728}{10.1111/cobi.13728}.

\bibitem[\citeproctext]{ref-dorcas_severe_2012}
Dorcas ME, Willson JD, Reed RN, Snow RW, Rochford MR, Miller MA, Meshaka WE, Andreadis PT, Mazzotti FJ, Romagosa CM, Hart KM. 2012. Severe mammal declines coincide with proliferation of invasive {Burmese} pythons in {Everglades} {National} {Park}. \emph{Proceedings of the National Academy of Sciences} 109:2418--2422. DOI: \href{https://doi.org/10.1073/pnas.1115226109}{10.1073/pnas.1115226109}.

\bibitem[\citeproctext]{ref-Durso2015}
Durso AM, Seigel RA. 2015. A {Snake} in the {Hand} is {Worth} 10,000 in the {Bush}. \emph{Journal of Herpetology} 49:503--506. DOI: \href{https://doi.org/10.1670/15-49-04.1}{10.1670/15-49-04.1}.

\bibitem[\citeproctext]{ref-gould_same_2023}
Gould E, Fraser H, Parker T, Nakagawa S, Griffith S, Vesk P, Fidler F, Abbey-Lee R, Abbott J, Aguirre L, Alcaraz C, Altschul D, Arekar K, Atkins J, Atkinson J, Barrett M, Bell K, Bello S, Berauer B, Bertram M, Billman P, Blake C, Blake S, Bliard L, Bonisoli-Alquati A, Bonnet T, Bordes C, Bose A, Botterill-James T, Boyd M, Boyle S, Bradfer-Lawrence T, Brand J, Brengdahl M, Bulla M, Bussière L, Camerlenghi E, Campbell S, Campos L, Caravaggi A, Cardoso P, Carroll C, Catanach T, Chen X, Chik HYJ, Choy E, Christie A, Chuang A, Chunco A, Clark B, Cox M, Cressman K, Crouch C, D'Amelio P, De Sousa A, Döbert T, Dobler R, Dobson A, Doherty T, Drobniak S, Duffy A, Dunn R, Dunning J, Eberhart-Hertel L, Elmore J, Elsherif M, English H, Ensminger D, Ernst U, Ferguson S, Ferreira-Arruda T, Fieberg J, Finch E, Fiorenza E, Fisher D, Forstmeier W, Fourcade Y, Francesca Santostefano F, Frank G, Freund C, Gandy S, Gannon D, García-Cervigón A, Géron C, Gilles M, Girndt A, Gliksman D, Goldspiel H, Gomes D, Goslee S, Gosnell J, Gratton P, Grebe N, Greenler S, Griffith D, Griffith F, Grossman J, Güncan A, Haesen S, Hagan J, Harrison N, Hasnain S, Havird J, Heaton A, Hsu B-Y, Iranzo E, Iverson E, Jimoh S, Johnson D, Johnsson M, Jorna J, Jucker T, Jung M, Kačergytė I, Ke A, Kelly C, Keogan K, Keppeler F, Killion A, Kim D, Kochan D, Korsten P, Kothari S, Kuppler J, Kusch J, Lagisz M, Larkin D, Larson C, Lauck K, Lauterbur M, Law A, Léandri-Breton D-J, Lievens E, Lima D, Lindsay S, Macphie K, Mair M, Malm L, Mammola S, Manhart M, Mäntylä E, Marchand P, Marshall B, Martin D, Martin J, Martin C, Martinig A, McCallum E, McNew S, Meiners S, Michelangeli M, Moiron M, Moreira B, Mortensen J, Mos B, Muraina T, Nelli L, Nilsonne G, Nolazco S, Nooten S, Novotny J, Olin A, Organ C, Ostevik K, Palacio F, Paquet M, Pascall D, Pasquarella V, Payo-Payo A, Pedersen K, Perez G, Perry K, Pottier P, Proulx M, Proulx R, Pruett J, Ramananjato V, Randimbiarison F, Razafindratsima O, Rennison D, Riva F, Riyahi S, Roast M, Rocha F, Roche D, Román-Palacios C, Rosenberg M, Ross J, Rowland F, Rugemalila D, Russell A, Ruuskanen S, Saccone P, Sadeh A, Salazar S, Sales K, Salmón P, Sanchez-Tojar A, Santos L, Schilling H, Schmidt M, Schmoll T, Schneider A, Schrock A, Schroeder J, Schtickzelle N, Schultz N, Scott D, Shapiro J, Sharma N, Shearer C, Sitvarin M, Skupien F, Slinn H, Smith J, Smith G, Sollmann R, Stack Whitney K, Still S, Stuber E, Sutton G, Swallow B, Taff C, Takola E, Tanentzap A, Thawley C, Tortorelli C, Trlica A, Turnell B, Urban L, Van De Vondel S, Van Oordt F, Vanderwel M, Vanderwel K, Vanderwolf K, Verrelli B, Vieira M, Vollering J, Walker X, Walter J, Waryszak P, Weaver R, Weller D, Whelan S, White R, Wolfson D, Wood A, Yanco S, Yen J, Youngflesh C, Zilio G, Zimmer C, Zitomer R, Villamil N, Tompkins E. 2023. Same data, different analysts: Variation in effect sizes due to analytical decisions in ecology and evolutionary biology. \emph{EcoEvoRxiv}. DOI: \href{https://doi.org/10.32942/X2GG62}{10.32942/X2GG62}.

\bibitem[\citeproctext]{ref-hodges_deadly_2021}
Hodges CW, Barnes CH, Patungtaro P, Strine CT. 2021. Deadly dormmate: {A} case study on {Bungarus} {candidus} living among a student dormitory with implications for human safety. \emph{Ecological Solutions and Evidence} 2. DOI: \href{https://doi.org/10.1002/2688-8319.12047}{10.1002/2688-8319.12047}.

\bibitem[\citeproctext]{ref-hodges_diurnal_2020}
Hodges CW, D'souza A, Jintapirom S. 2020. Diurnal observation of a {Malayan} {Krait} {Bungarus} candidus ({Reptilia}: {Elapidae}) feeding inside a building in {Thailand}. \emph{Journal of Threatened Taxa} 12:15947--15950. DOI: \href{https://doi.org/10.11609/jott.5746.12.8.15947-15950}{10.11609/jott.5746.12.8.15947-15950}.

\bibitem[\citeproctext]{ref-hodges_malayan_2022}
Hodges CW, Marshall BM, Hill JG, Strine CT. 2022. Malayan kraits ({Bungarus} candidus) show affinity to anthropogenic structures in a human dominated landscape. \emph{Scientific Reports} 12:7139. DOI: \href{https://doi.org/10.1038/s41598-022-11255-z}{10.1038/s41598-022-11255-z}.

\bibitem[\citeproctext]{ref-huntingtonklein_influence_2021}
Huntington‐Klein N, Arenas A, Beam E, Bertoni M, Bloem JR, Burli P, Chen N, Grieco P, Ekpe G, Pugatch T, Saavedra M, Stopnitzky Y. 2021. The influence of hidden researcher decisions in applied microeconomics. \emph{Economic Inquiry} 59:944--960. DOI: \href{https://doi.org/10.1111/ecin.12992}{10.1111/ecin.12992}.

\bibitem[\citeproctext]{ref-jennions_relationships_2002}
Jennions MD, Møller AP. 2002. Relationships fade with time: A meta-analysis of temporal trends in publication in ecology and evolution. \emph{Proceedings of the Royal Society of London. Series B: Biological Sciences} 269:43--48. DOI: \href{https://doi.org/10.1098/rspb.2001.1832}{10.1098/rspb.2001.1832}.

\bibitem[\citeproctext]{ref-Jones_supposed_2020}
Jones MD, Crane MS, Silva IMS, Artchawakom T, Waengsothorn S, Suwanwaree P, Strine CT, Goode M. 2020. Supposed snake specialist consumes monitor lizards: Diet and trophic implications of king cobra feeding ecology. \emph{Ecology}. DOI: \href{https://doi.org/10.1002/ecy.3085}{10.1002/ecy.3085}.

\bibitem[\citeproctext]{ref-jones_how_2022}
Jones MD, Marshall BM, Smith SN, Crane M, Silva I, Artchawakom T, Suwanwaree P, Waengsothorn S, Wüster W, Goode M, Strine CT. 2022. How do {King} {Cobras} move across a major highway? {Unintentional} wildlife crossing structures may facilitate movement. \emph{Ecology and Evolution} 12. DOI: \href{https://doi.org/10.1002/ece3.8691}{10.1002/ece3.8691}.

\bibitem[\citeproctext]{ref-joo_recent_2022}
Joo R, Picardi S, Boone ME, Clay TA, Patrick SC, Romero-Romero VS, Basille M. 2022. Recent trends in movement ecology of animals and human mobility. \emph{Movement Ecology} 10:26. DOI: \href{https://doi.org/10.1186/s40462-022-00322-9}{10.1186/s40462-022-00322-9}.

\bibitem[\citeproctext]{ref-kays_movebank_2022}
Kays R, Davidson SC, Berger M, Bohrer G, Fiedler W, Flack A, Hirt J, Hahn C, Gauggel D, Russell B, Kölzsch A, Lohr A, Partecke J, Quetting M, Safi K, Scharf A, Schneider G, Lang I, Schaeuffelhut F, Landwehr M, Storhas M, Schalkwyk L, Vinciguerra C, Weinzierl R, Wikelski M. 2022. The {Movebank} system for studying global animal movement and demography. \emph{Methods in Ecology and Evolution} 13:419--431. DOI: \href{https://doi.org/10.1111/2041-210X.13767}{10.1111/2041-210X.13767}.

\bibitem[\citeproctext]{ref-kelly_rate_2019}
Kelly CD. 2019. Rate and success of study replication in ecology and evolution. \emph{PeerJ} 7:e7654. DOI: \href{https://doi.org/10.7717/peerj.7654}{10.7717/peerj.7654}.

\bibitem[\citeproctext]{ref-Knierim2017a}
Knierim T, Barnes CH, Hodges C. 2017. BUNGARUS FASCIATUS banded krait. DIET SCAVENGING. \emph{Herpetological Review} 48:215--218.

\bibitem[\citeproctext]{ref-knierim_spatial_2019}
Knierim T, Strine CT, Suwanwaree P, Hill III JG. 2019. Spatial ecology study reveals nest attendance and habitat preference of banded kraits ({Bungarus} fasciatus). \emph{Herpetological Bulletin}:6--13. DOI: \href{https://doi.org/10.33256/hb150.613}{10.33256/hb150.613}.

\bibitem[\citeproctext]{ref-kuch_notes_2001}
Kuch U. 2001. {NOTES} {ON} {THE} {DIET} {OF} {THE} {MALAYAN} {KRAIT}, {BUNGARUS} {CANDIDUS} ({LINNAEUS}, 1758). \emph{Herpetological Bulletin} 75.

\bibitem[\citeproctext]{ref-looareesuwan_factors_1988}
Looareesuwan S, Viravan C, Warrell DA. 1988. Factors contributing to fatal snake bite in the rural tropics: Analysis of 46 cases in {Thailand}. \emph{Transactions of The Royal Society of Tropical Medicine and Hygiene} 82:930--934. DOI: \href{https://doi.org/10.1016/0035-9203(88)90046-6}{10.1016/0035-9203(88)90046-6}.

\bibitem[\citeproctext]{ref-marshall_no_2020}
Marshall BM, Crane M, Silva I, Strine CT, Jones MD, Hodges CW, Suwanwaree P, Artchawakom T, Waengsothorn S, Goode M. 2020. No room to roam: {King} {Cobras} reduce movement in agriculture. \emph{Movement Ecology} 8:33. DOI: \href{https://doi.org/10.1186/s40462-020-00219-5}{10.1186/s40462-020-00219-5}.

\bibitem[\citeproctext]{ref-Marshall2018}
Marshall BM, Strine CT, Jones MD, Artchawakom T, Silva I, Suwanwaree P, Goode M. 2019. Space fit for a king: Spatial ecology of king cobras ({Ophiophagus} hannah) in {Sakaerat} {Biosphere} {Reserve}, {Northeastern} {Thailand}. \emph{Amphibia-Reptilia} 40:163--178. DOI: \href{https://doi.org/10.1163/15685381-18000008}{10.1163/15685381-18000008}.

\bibitem[\citeproctext]{ref-Marshall2018b}
Marshall BM, Strine CT, Jones MD, Theodorou A, Amber E, Waengsothorn S, Suwanwaree P, Goode M. 2018. Hits {Close} to {Home}: {Repeated} {Persecution} of {King} {Cobras} ({Ophiophagus} hannah) in {Northeastern} {Thailand}. \emph{Tropical Conservation Science} 11:194008291881840. DOI: \href{https://doi.org/10.1177/1940082918818401}{10.1177/1940082918818401}.

\bibitem[\citeproctext]{ref-Miranda2016}
Miranda EBP, Ribeiro- RP, Strüssmann C. 2016. The ecology of human-anaconda conflict: A study using internet videos. \emph{Tropical Conservation Science} 9:43--77. DOI: \href{https://doi.org/10.1177/194008291600900105}{10.1177/194008291600900105}.

\bibitem[\citeproctext]{ref-nakagawa_replicating_2015}
Nakagawa S, Parker TH. 2015. Replicating research in ecology and evolution: Feasibility, incentives, and the cost-benefit conundrum. \emph{BMC Biology} 13:88. DOI: \href{https://doi.org/10.1186/s12915-015-0196-3}{10.1186/s12915-015-0196-3}.

\bibitem[\citeproctext]{ref-noonan_effects_2020}
Noonan MJ, Fleming CH, Tucker MA, Kays R, Harrison A, Crofoot MC, Abrahms B, Alberts SC, Ali AH, Altmann J, Antunes PC, Attias N, Belant JL, Beyer DE, Bidner LR, Blaum N, Boone RB, Caillaud D, Paula RC, Torre JA la, Dekker J, DePerno CS, Farhadinia M, Fennessy J, Fichtel C, Fischer C, Ford A, Goheen JR, Havmøller RW, Hirsch BT, Hurtado C, Isbell LA, Janssen R, Jeltsch F, Kaczensky P, Kaneko Y, Kappeler P, Katna A, Kauffman M, Koch F, Kulkarni A, LaPoint S, Leimgruber P, Macdonald DW, Markham AC, McMahon L, Mertes K, Moorman CE, Morato RG, Moßbrucker AM, Mourão G, O'Connor D, Oliveira‐Santos LGR, Pastorini J, Patterson BD, Rachlow J, Ranglack DH, Reid N, Scantlebury DM, Scott DM, Selva N, Sergiel A, Songer M, Songsasen N, Stabach JA, Stacy‐Dawes J, Swingen MB, Thompson JJ, Ullmann W, Vanak AT, Thaker M, Wilson JW, Yamazaki K, Yarnell RW, Zieba F, Zwijacz‐Kozica T, Fagan WF, Mueller T, Calabrese JM. 2020. Effects of body size on estimation of mammalian area requirements. \emph{Conservation Biology}:cobi.13495. DOI: \href{https://doi.org/10.1111/cobi.13495}{10.1111/cobi.13495}.

\bibitem[\citeproctext]{ref-peterson_self-correction_2021}
Peterson D, Panofsky A. 2021. Self-correction in science: {The} diagnostic and integrative motives for replication. \emph{Social Studies of Science}.

\bibitem[\citeproctext]{ref-Pochanugool1998}
Pochanugool C, Wildde H, Bhanganada K, Chanhome L, Cox MJ, Chaiyabutr N, Sitprija V. 1998. \href{https://www.ncbi.nlm.nih.gov/pubmed/9597849}{Venomous snakebite in {Thailand}. {II}: {Clinical} experience}. \emph{Mil Med} 163:318--323.

\bibitem[\citeproctext]{ref-portugal_externally_2022}
Portugal SJ, White CR. 2022. Externally attached biologgers cause compensatory body mass loss in birds. \emph{Methods in Ecology and Evolution} 13:294--302. DOI: \href{https://doi.org/10.1111/2041-210X.13754}{10.1111/2041-210X.13754}.

\bibitem[\citeproctext]{ref-Rao2013}
Rao C, Talukdar G, Choudhury BC, Shankar PG, Whitaker R, Goode M. 2013. Habitat use of {King} {Cobra} ({Ophiophagus} hannah) in a heterogeneous landscape matrix in the tropical forests of the {Western} {Ghats} , {India}. \emph{Hamadryad} 36:69--79.

\bibitem[\citeproctext]{ref-robstad_impact_2021}
Robstad CA, Lodberg-Holm HK, Mayer M, Rosell F. 2021. The impact of bio-logging on body weight change of the {Eurasian} beaver. \emph{PLOS ONE} 16:e0261453. DOI: \href{https://doi.org/10.1371/journal.pone.0261453}{10.1371/journal.pone.0261453}.

\bibitem[\citeproctext]{ref-salis_how_2021}
Salis A, Lena J-P, Lengagne T. 2021. How {Subtle} {Protocol} {Choices} {Can} {Affect} {Biological} {Conclusions}: {Great} {Tits}' {Response} to {Allopatric} {Mobbing} {Calls}. \emph{Animal Behavior and Cognition} 8:152--165. DOI: \href{https://doi.org/10.26451/abc.08.02.05.2021}{10.26451/abc.08.02.05.2021}.

\bibitem[\citeproctext]{ref-Shankar2013a}
Shankar PG, Singh A, Ganesh SR, Whitaker R. 2013. Factors influencing human hostility to {King} {Cobras} ({Ophiophagus} hannah) in the {Western} {Ghats} of {India}. \emph{Hamadryad} 36:91--100.

\bibitem[\citeproctext]{ref-silberzahn_many_2018}
Silberzahn R, Uhlmann EL, Martin DP, Anselmi P, Aust F, Awtrey E, Bahník Š, Bai F, Bannard C, Bonnier E, Carlsson R, Cheung F, Christensen G, Clay R, Craig MA, Dalla Rosa A, Dam L, Evans MH, Flores Cervantes I, Fong N, Gamez-Djokic M, Glenz A, Gordon-McKeon S, Heaton TJ, Hederos K, Heene M, Hofelich Mohr AJ, Högden F, Hui K, Johannesson M, Kalodimos J, Kaszubowski E, Kennedy DM, Lei R, Lindsay TA, Liverani S, Madan CR, Molden D, Molleman E, Morey RD, Mulder LB, Nijstad BR, Pope NG, Pope B, Prenoveau JM, Rink F, Robusto E, Roderique H, Sandberg A, Schlüter E, Schönbrodt FD, Sherman MF, Sommer SA, Sotak K, Spain S, Spörlein C, Stafford T, Stefanutti L, Tauber S, Ullrich J, Vianello M, Wagenmakers E-J, Witkowiak M, Yoon S, Nosek BA. 2018. Many {Analysts}, {One} {Data} {Set}: {Making} {Transparent} {How} {Variations} in {Analytic} {Choices} {Affect} {Results}. \emph{Advances in Methods and Practices in Psychological Science} 1:337--356. DOI: \href{https://doi.org/10.1177/2515245917747646}{10.1177/2515245917747646}.

\bibitem[\citeproctext]{ref-smith_native_2021}
Smith SN, Jones MD, Marshall BM, Waengsothorn S, Gale GA, Strine CT. 2021. Native {Burmese} pythons exhibit site fidelity and preference for aquatic habitats in an agricultural mosaic. \emph{Scientific Reports} 11:7014. DOI: \href{https://doi.org/10.1038/s41598-021-86640-1}{10.1038/s41598-021-86640-1}.

\bibitem[\citeproctext]{ref-searo_regional_office_for_the_south_east_asia_rgo_guidelines_2016}
South East Asia (RGO) SRO for the, Asia WS-E. 2016. \emph{\href{https://www.who.int/publications/i/item/9789290225300}{Guidelines for the management of snakebites, 2nd edition}}. WHO.

\bibitem[\citeproctext]{ref-Stuart2012}
Stuart B, Wogan G, Grismer L, Auliya M, Inger RF, Lilley R, Chan-Ard T, Thy N, Nguyen TQ, Srinivasulu C, Jelić D. 2012. \href{http://dx.doi.org/10.2305/IUCN.UK.2012-\%201.RLTS.T177540A1491874.en\%0ACopyright:}{Ophiophagus hannah, {King} {Cobra}}. \emph{The IUCN Red List of Threatened Species 2012}.

\bibitem[\citeproctext]{ref-stuber_spatial_2022}
Stuber EF, Carlson BS, Jesmer BR. 2022. Spatial personalities: A meta-analysis of consistent individual differences in spatial behavior. \emph{Behavioral Ecology} 33:477--486. DOI: \href{https://doi.org/10.1093/beheco/arab147}{10.1093/beheco/arab147}.

\bibitem[\citeproctext]{ref-tomotani_great_2021}
Tomotani BM, Muijres FT, Johnston B, Jeugd HP, Naguib M. 2021. Great tits do not compensate over time for a radio‐tag‐induced reduction in escape‐flight performance. \emph{Ecology and Evolution} 11:16600--16617. DOI: \href{https://doi.org/10.1002/ece3.8240}{10.1002/ece3.8240}.

\bibitem[\citeproctext]{ref-Viravan1992}
Viravan C, Looareesuwan S, Kosakam W, Wuthiekanun V, McCarthy CJ, Stimson AF, Bunnag D, Harinasuta T, Warrell DA. 1992. A national hospital-based survey of snakes responsible for bites in {Thailand}. \emph{Transactions of the Royal Society of Tropical Medicine and Hygiene} 86:100--106. DOI: \href{https://doi.org/10.1016/0035-9203(92)90463-M}{10.1016/0035-9203(92)90463-M}.

\bibitem[\citeproctext]{ref-Weatherhead2004}
Weatherhead PJ, Blouin-Demers G. 2004. Long-term effects of radiotelemetry on black ratsnakes. \emph{Wildlife Society Bulletin} 32:900--906. DOI: \href{https://doi.org/10.2193/0091-7648(2004)032\%5B0900:LEOROB\%5D2.0.CO;2}{10.2193/0091-7648(2004)032{[}0900:LEOROB{]}2.0.CO;2}.

\bibitem[\citeproctext]{ref-Whitaker2004}
Whitaker R, Captain A. 2004. King {Cobra} {Ophiophagous} hannah. In: \emph{Snakes of {India} - {A} {Field} {Guide}}. Chennai, India: Draco Books, 384--385.

\end{CSLReferences}

\end{document}
