
\documentclass[10pt,a4paper]{article}
\usepackage{f1000_styles}

%% Default: numerical citations
% \usepackage[numbers]{natbib}

%% Uncomment this lines for superscript citations instead
% \usepackage[super]{natbib}

%% Uncomment these lines for author-year citations instead
% \usepackage[round]{natbib}
% \let\cite\citep

%% lines required to use a CSL style for references
\newlength{\cslhangindent}
\setlength{\cslhangindent}{1.5em}
\newlength{\csllabelwidth}
\setlength{\csllabelwidth}{3em}
\newlength{\cslentryspacingunit} % times entry-spacing
\setlength{\cslentryspacingunit}{\parskip}
\newenvironment{CSLReferences}[2] % #1 hanging-ident, #2 entry spacing
 {% don't indent paragraphs
  \setlength{\parindent}{0pt}
  % turn on hanging indent if param 1 is 1
  \ifodd #1
  \let\oldpar\par
  \def\par{\hangindent=\cslhangindent\oldpar}
  \fi
  % set entry spacing
  \setlength{\parskip}{#2\cslentryspacingunit}
 }%
 {}
\usepackage{calc}
\newcommand{\CSLBlock}[1]{#1\hfill\break}
\newcommand{\CSLLeftMargin}[1]{\parbox[t]{\csllabelwidth}{#1}}
\newcommand{\CSLRightInline}[1]{\parbox[t]{\linewidth - \csllabelwidth}{#1}\break}
\newcommand{\CSLIndent}[1]{\hspace{\cslhangindent}#1}

%% lines to get the code chunks working

%% lines to enable bulletpoints in a new notation style
\providecommand{\tightlist}{%
  \setlength{\itemsep}{0pt}\setlength{\parskip}{0pt}}

\begin{document}
\pagestyle{fancy}

\title{Applying a Multiverse to Snake Habitat Selection}
\author[1]{Benjamin Michael Marshall*}
\author[1]{Alexander Bradley Duthie**}
\affil[1]{Biological and Environmental Sciences, Faculty of Natural Sciences, University of Stirling, Stirling, FK9 4LA, Scotland, UK}

\affil[*]{\href{mailto:benjaminmichaelmarshall@gmail.com}{\nolinkurl{benjaminmichaelmarshall@gmail.com}}}
\affil[**]{\href{mailto:alexander.duthie@stir.ac.uk}{\nolinkurl{alexander.duthie@stir.ac.uk}}}

\maketitle
\thispagestyle{fancy}

\begin{abstract}

abc

\end{abstract}

\section*{Keywords}

Movement ecology, step selection function, habitat preference, habitat selection, animal movement, multiverse, research choice, researcher degrees for freedom,

\clearpage
\pagestyle{fancy}

\hypertarget{introduction}{%
\section{Introduction}\label{introduction}}

A key component of science is the continual reassessment of past work and findings (\protect\hyperlink{ref-alberts_self-correction_2015}{Alberts et al., 2015}).
Whether that takes the form of direct replications aiming to discover how exactly how reliable previous work is, or more integrative approaches testing the edges of previous findings' generalisability and retesting questions in different study systems (\protect\hyperlink{ref-nakagawa_replicating_2015}{Nakagawa \& Parker, 2015}; \protect\hyperlink{ref-peterson_self-correction_2021}{Peterson \& Panofsky, 2021}).

Reassessments and replications --regardless of their position on the direct-quasi continuum-- can aid the formal and organic self-correcting process of science.
Initial findings set the stage for subsequent work, building momentum that can accelerate progress, but also be difficult to redirect if the initial impetus was misdirected (\protect\hyperlink{ref-jennions_relationships_2002}{Jennions \& Møller, 2002}; \protect\hyperlink{ref-barto_dissemination_2012}{Barto \& Rillig, 2012}) .
Therefore, checking and confirming results early is important; we can see this principle recognised in the peer review system itself.

Checking previous findings through replication can be come more difficult in systems with high task uncertainty.
High task uncertainty systems --those that manifest high levels of uncontrollable stochasticity-- may make direct diagnostic replications impractical or impossible, and render the evidence from quasi-replications weaker (\protect\hyperlink{ref-peterson_self-correction_2021}{Peterson \& Panofsky, 2021}).
Ecological systems can be considered as generating high task uncertainty, with many interconnected elements, and when studying wild systems many of those elements are uncontrollable.

With ecological systems, such complexity and the difficulty to control an experiment makes the direct replications costly, potentially explaining their rarity (\protect\hyperlink{ref-kelly_rate_2019}{Kelly, 2019}).
When studying wild animals with a level of direct intervention, repeating experiments/studies might be unethical due to the well-being costs (\protect\hyperlink{ref-Weatherhead2004}{Weatherhead \& Blouin-Demers, 2004}; \protect\hyperlink{ref-robstad_impact_2021}{Robstad et al., 2021}; \protect\hyperlink{ref-tomotani_great_2021}{Tomotani et al., 2021}; \protect\hyperlink{ref-portugal_externally_2022}{Portugal \& White, 2022}; \protect\hyperlink{ref-altobelli_methods_2022}{Altobelli et al., 2022}).

When faced with limited options for direct replications, an alternative, albeit not a replacement, would be to re-examine existing datasets.
Pooling old and new datasets, and reanalysising them may provide insights into broader generalisations.

In some cases older data may have been collected/recorded in away that enables completely fresh analysis (\protect\hyperlink{ref-kays_movebank_2022}{Kays et al., 2022}).
As methodologies develop, conceptualisations change, and computational power increases, new avenues for examining the same data may materialise (e.g., \protect\hyperlink{ref-noonan_effects_2020}{Noonan et al., 2020}).
As these new methods are developed and applied, we may see the conclusions based upon those data change.
There are a growing number of examples demonstrating that the analysis approach can alter the results (\protect\hyperlink{ref-salis_how_2021}{Salis, Lena \& Lengagne, 2021}; \protect\hyperlink{ref-desbureaux_subjective_2021}{Desbureaux, 2021}), and that the researchers themselves can be a key source of variation in analysis and results (\protect\hyperlink{ref-silberzahn_many_2018}{Silberzahn et al., 2018}; \protect\hyperlink{ref-huntingtonklein_influence_2021}{Huntington‐Klein et al., 2021}; \protect\hyperlink{ref-gould_same_2023}{Gould et al., 2023}).
These examples elegantly show the possible extent of technical uncertainty present in some systems .

Not all disciplines have explored the sources of uncertainty in findings equally.
Prudence would push for examination of uncertainty in all its forms, in particular for fields that already tackle high levels of uncertainty originating from a wild study system.
Movement ecology could be argued to exemplify such a field.
Animals are complex, existing in complex wild ecosystems, with individuality and personality (\protect\hyperlink{ref-stuber_spatial_2022}{Stuber, Carlson \& Jesmer, 2022}).
Depending on the research question, controls in movement ecology can be difficult to achieve, and replications difficult to justify given the strict ethical limitations on interventionist study.
Movement ecology has also seized the opportunities presented by technological developments, enabling higher resolution tracking of animal movements (e.g., GPS tracking) and more sophisticated analysis that can integrate the high dimensional data {[}e.g., x-y coordinates, time, acceleration, individual, other covariates of interest; Joo et al. (\protect\hyperlink{ref-joo_recent_2022}{2022}){]}.

Personality and the repeatability of behaviours presents a key component to the uncertainty or variation when attempting to generalise.
However, here we turn to the technical uncertainty, the uncertainty originating from the researcher and how they approach the data.
Previous many analyst projects highlight the potential for analyst-side variation (\protect\hyperlink{ref-silberzahn_many_2018}{Silberzahn et al., 2018}; \protect\hyperlink{ref-huntingtonklein_influence_2021}{Huntington‐Klein et al., 2021}; \protect\hyperlink{ref-gould_same_2023}{Gould et al., 2023}), and previous multiverse explorations of movement ecology methods highlight the variation potentially presented within a synthetic movement dataset .
Here we take the multiverse approach further by applying it to a number of real case studies with the aim of exploration whether different analysis approaches could/would have altered the final general conclusions.

We selected a quartet of separate but connected movement ecology studies that attempt to disentangle the habitat selection exhibited by snakes in north-eastern Thailand.
All four cases focus on snakes that come into conflict with humans to some extent, either because of the risks poses from their venom (king cobra, Malayan krait, banded krait), or because of their appetite for domestic livestock (Burmese python).
In all cases the habitat selection results could be used to guide snake conservation efforts, as well as interventions into human behaviour to mitigate human-snake conflict.
With these general goals in mind, we re-examine the movement datasets using a multiverse of habitat selection analysis pathways to reveal whether the same data could lead to different conclusions.

\hypertarget{methods}{%
\section{Methods}\label{methods}}

\hypertarget{study-location}{%
\subsection{Study Location}\label{study-location}}

All four case studies occurred in north eastern Thailand, within Nakhon Ratchasima province.
Three case studies (king cobra, Burmese python, banded krait) were conducted within the Sakaerat Biosphere Reserve.
The reserve comprises of three zones of management: core, buffer, and transitional.
The core is largely primary forest; the buffer surrounds the core and is comprised of forest regeneration efforts, whereas the transitional zone allows more development resulting in a mix of agriculture, settlements, and plantation forest.
Bisecting the transitional zone, and running adjacent to the protected forest areas is a four-lane highway connecting the city of Nakhon Ratchasima to Bangkok.
The case study (Malayan krait) not in the Sakaerat Biosphere Reserve was undertaken nearer to Nakhon Ratchasima proper, on the Suranaree University of Technology.
The university campus is a mix of scrub forest, open lawn, university buildings, and homes.

\hypertarget{study-species-and-hypotheses}{%
\subsection{Study Species and Hypotheses}\label{study-species-and-hypotheses}}

Snakes can be difficult to detect in wild scenarios , forcing a wider and more opportunistic suite of methods to gather adequate sample sizes.
In all the chosen case studies snakes were obtained for study using trapping arrays, active surveying, and notifications from locals.
The local notifications often arose from snakes entering human settlements, and a desire for the snake to be removed.

The four case studies cover four snake species, each with their own ecology and movements.

\hypertarget{king-cobra}{%
\subsubsection{King Cobra}\label{king-cobra}}

Marshall et al. (\protect\hyperlink{ref-Marshall2018}{2019}) and Marshall et al. (\protect\hyperlink{ref-marshall_no_2020}{2020}) are concerned with king cobras (\emph{Ophiophagus hannah}).
King cobras are a large (tracked individuals between ), diurnal, active foraging snake species that depredate snakes and monitor lizards .
While considered a predominately forest dwelling species , they are know to make use of more human altered areas , which can lead to frequent in human-snake conflict .
The extremely low occurrence of king cobra bites in Thailand mean that instances of human-snake conflict are primarily a conservation concern as opposed to human health .

Marshall et al. (\protect\hyperlink{ref-Marshall2018}{2019}) does not conclude on an actual selection, instead highlighting the king cobras excursions out of the protected forest.
Marshall et al. (\protect\hyperlink{ref-marshall_no_2020}{2020}) looks more specifically at selection, highlighting the importance of semi-natural areas that occupy the banks of irrigation canals and intersection the agricultural areas surrounding the protected forest.
Therefore, we will pool both datasets and examine two non-mutually exclusion hypotheses that can be examined through a unified model.

H\textsubscript{OpHa1}: King Cobras select for semi-natural habitat

H\textsubscript{OpHa2}: King Cobras select for forest habitat

\hypertarget{burmese-python}{%
\subsubsection{Burmese Python}\label{burmese-python}}

Smith et al. (\protect\hyperlink{ref-smith_native_2021}{2021}) describe Burmese python (\emph{Python bivittatus}) habitat selection and movement.
Burmese pythons are large (tracked individuals between ), ambush predators capable of tacking prey \_\_\_\% larger than their body mass .
The flexibility in regards to prey size means they are inevitably draw into conflict with humans over livestock, a pattern mirrored across the globe for large snakes .

The conclusions of Smith et al. (\protect\hyperlink{ref-smith_native_2021}{2021}) on python habitat selection are not dissimilar to those made on king cobras, with an active selection for areas near water.
The land classification used in Smith et al. (\protect\hyperlink{ref-smith_native_2021}{2021}) was slightly different to Marshall et al. (\protect\hyperlink{ref-marshall_no_2020}{2020}), grouping semi-natural areas with larger water bodies (e.g., agricultural ponds).

H\textsubscript{PyBi1}: Burmese Pythons select for areas near water.

\hypertarget{malayan-krait}{%
\subsubsection{Malayan Krait}\label{malayan-krait}}

Hodges et al. (\protect\hyperlink{ref-hodges_malayan_2022}{2022}) examine a smaller species, the Malayan krait (\emph{Bungarus candidus}).
The Malayan kraits tracked were between \_\_ and \_\_m snout to vent, and all lived on a university campus.
Malayan kraits like many elapids, have a potent and medically significant venom; bites of Malayan kraits can be fatal.
They are nocturnal and actively foraging, with a suggested preference for frogs and small rodents .

Unlike the other case studies, Hodges et al. (\protect\hyperlink{ref-hodges_malayan_2022}{2022}) is undertaken in a more urban environment.
The scale of the Malayan krait movements meant the study was conducted at a finer spatial scale; habitat types are therefore more finely separated (e.g., buildings vs settlements).
The overall conclusions highlight a number of habitat types that potentially being selected for, and in opposition an avoidance of open areas.

H\textsubscript{BuCa1}: Malayan Kraits select for buildings, settlements, and natural areas.

\hypertarget{banded-krait}{%
\subsubsection{Banded Krait}\label{banded-krait}}

Knierim (\protect\hyperlink{ref-knierim_spatial_2019}{2019}) looked at a larger krait species, the banded krait (\emph{Bungarus fasciatus}).
Like its smaller cousin the banded krait is also a nocturnal active forager, with a potent venom.
The banded krait is heavier bodies and grows to longer lengths, tracked individuals ranging from snout to vent length
However, unlike the Malayan krait, the banded krait appears less tolerant of human disturbance in this region of Thailand and tends to have a more ophiophagus diet .

Banded kraits were entirely located in agricultural land, and like the other krait had movements more conducive to finer habitat classifications.
For example, field margins were found as a key nesting site .
Knierim (\protect\hyperlink{ref-knierim_spatial_2019}{2019}) shows that importance is reflected in the movements and habitat selection, as banded kraits follow the linear water or field margin features as opposed to the wider more exposed field areas.

H\textsubscript{BuFa1}: Banded Kraits select for waterways and field edges.

\hypertarget{multiverse-construction}{%
\subsection{Multiverse Construction}\label{multiverse-construction}}

To explore the variation that could arise when testing the above hypotheses, we constructed a multiverse of analytical choices.
A multiverse is tree of branching paths, where decisions made during analysis spawn multiple branches, each with alternative answers.
Previous work has reveal the importance of data quantity over analysis decisions in determining habitat selection , but we are still left with sizeable variation in final answers --and perhaps more importantly answers that offer contradictory conclusions.
The variation in results are similarly present in more practical demonstrations, such as many analysts projects .

Our multiverse starts with two alternative ways of defining the habitats.
Each case study has a map of available habitats, defined as categories.
One approach we used is to examine the habitat selection by retaining these categories, only simplifying them to fit the hypotheses.
This simplification took the habitat(s) of interest and codes them as 1, whereas all other habitats previously indicated in the studies to be avoided/not-selected for are coded as 0.
For example, in the case of king cobras, forest and semi-natural areas were classed as 1, and all other types classed as 0.
We then used the 0/1 classification as a predictor in the habitat selection analyses.
We undertook this simplification to facilitate the repeated use of generic code in the multiverse.
The second approach involved converting the simplified categorical habitat types into continuous rasters, where each cell described the distance to a given habitat type.
We inverted the distances, so that a positive effect in the model mean a positive selection towards a habitat type (resulting in a more intuitive final output).

The habitat selection methods we explored can be broadly grouped into three types:
- area based RSF,
- step based SSF and Poisson
- Weighted resource selection based

The first category require a define area of availability, for which random available points can be drawn.
To explore the impact of the definition of available, we created a number of different polygons surrounding the tracking data locations.
We created available areas using Minimum Convex Polygons (MCP), Kernel Density Estimates (KDE: using the reference smoothing bandwidth {[}href{]}), and Autocorrelated Kernel Density Estimates (aKDE).
MCP-
KDE-
aKDE-
All three methods require a selection of an outer most contour to define the edge of the available habitat, we varied this outer edge including analysis of 90, 95, and 99\% contours.
For the aKDE areas, we selected the point estimate connected to the contour percentage, ignoring the 95\% confidence intervals associated with the estimate.
Once an available area was defined, we generated points within the area to extract the available habitat.
We varied the point generation process, either purely random or stratified across the overall area.
In addition to the point generation method, we varied the number of points created from \_\_\_ to \_\_\_.
As each individual had their own available area, we explored Type III habitat selection; however, as the question could be equally explored via a Type II design, we merged all available areas into a landscape level area of available habitat.
Overall, for RSF we explored the impact of area method, area contour, point generation method, number of generated points, and type design.

The second group, instead of using an available area, use available steps randomly generated for each time step.
For these step-based methods a different suite of choices were explored.
First is the number of random steps generated per known location, we ranged this from \_\_\_ to \_\_\_.
To generate those steps we draw values from distributions, the choice of those distributions make up the next two choices.
For the random step lengths we looked at impact of using Gamma and Exponential distributions; whereas for the turn angles we looked at Von Mises and Uniform distributions.
Once the random available locations had been generated, we explore an additional choice regarding the model formula: whether to have the step lengths and turn angles interact with the habitat.
Termed integrated step-selection, the inclusion of the step lengths and turn angles is meant to aid the acquisition of less biased estimates of habitat selection.
However, the impacts of the inclusion have differing impacts when using step-selection or a poisson model ; we include it here to explore how dramatic that difference can be.

Finally, we assessed Weighted Resource Selection Function .
Weighted Resource Selection Functions are the newest of the methods examined here.
Instead of relying on available areas they\ldots{} MORE HERE

\hypertarget{notes}{%
\section{notes}\label{notes}}

The papers below use distance from feature in the models, we could add an exploration of binary habitats too.
Overall simplify the models just to the habitats of interest.
The python paper has a contrasting grouping of what makes water habitats, do we need to explore alternative groupings of land use types - suspect that we don't but instead use the paper specific definitions (except for the 2018 paper because the number of habitat types is high and ill defined).

Any paper using SSF methods we should summarise to the population level to get an estimate we can plot alongside the multiverse answers.
Won't be a perfect comparison, but least on a similar scale.
So we can get a naive mean of SSF preference for given habitat, and in the case of the population ones we can just use that estimate + CIs.

Each species will get model ran on three landscape configurations:
- original, using whatever method/classification system used, but we will only extract the estimates of direct interest to the hypotheses. Model formula will include all habitat types. This will also provide a means of reproducing the original answers for direct comparison to the other estimates from other decisions pathways. 2018 doesn't have this, so will use 2020 LU data.
- targeted continuous, a distance to landscape feature approach but refined only to landscapes that matter (e.g., inverted distance to semi-nat vs inverted distance to everything else).
- targeted binary, simply hypothesised good habitat versus everything else.

\hypertarget{king-cobra-1}{%
\subsubsection{King Cobra}\label{king-cobra-1}}

Marshall et al. (\protect\hyperlink{ref-Marshall2018}{2019}); Marshall et al. (\protect\hyperlink{ref-marshall_no_2020}{2020})
OPHA
2018 paper doesn't conclude a habitat preference, more focusing on the fact that they do not remain within the forest.
The methods are a major limitation in that regard.
Best course of action is likely to look at semi-nat areas and forest, and examine the validity of both claims.
Using the results from the iSSF in figure 4 primarily, clear semi-nat preference that works for a clean re-testable hypo, forest preference can be additional motivation to make the 2018 paper worth exploring too.
Didn't actually implement any population level summary, but we are testing general conclusions here.
As the other papers have excluded individuals, what if we do the same here for the 2018 snakes, so we have something just targeting the 2020 paper and then something more general? Would make the targets pipeline more intuitive as all species would have this decision.
Hypothesis 1: King Cobras show preference for semi-natural habitat
Hypothesis 2: King Cobras show preference for forest habitat

\hypertarget{burmese-python-1}{%
\subsubsection{Burmese Python}\label{burmese-python-1}}

Smith et al. (\protect\hyperlink{ref-smith_native_2021}{2021})
PYBI
excluded an individual for the pop-level stuff - possible choice to explore?
Figure 4 shows the clear population pattern preferring water (water bodies and semi-nat areas). Figure 5 shows the preference on an individual level also.
Hypothesis: Burmese Pythons select for areas near water.

\hypertarget{malayan-krait-1}{%
\subsubsection{Malayan Krait}\label{malayan-krait-1}}

Hodges et al. (\protect\hyperlink{ref-hodges_malayan_2022}{2022})
BUCA
Had excluded individuals - one simply doesn't have enough data; the other should be included(?) but it remained in one habitat type - possible choice to explore?
Hypothesis: Malayan Kraits select for buildings, settlements, and natural areas.

\hypertarget{banded-krait-1}{%
\subsubsection{Banded Krait}\label{banded-krait-1}}

Knierim (\protect\hyperlink{ref-knierim_spatial_2019}{2019})
BUFA
Need to see if we can get the LU data.
Hypothesis: Banded Kraits select for waterways and field edges.

\hypertarget{r-openoptions-optionscompletelist---readrdshereheredata-optionscompletelist.rds}{%
\section{\texorpdfstring{\texttt{\{r\ openOptions\}\ \#\ optionsCompleteList\ \textless{}-\ readRDS(here::here("data",\ "optionsCompleteList.rds"))\ \#}}{\{r openOptions\} \# optionsCompleteList \textless- readRDS(here::here("data", "optionsCompleteList.rds")) \#}}\label{r-openoptions-optionscompletelist---readrdshereheredata-optionscompletelist.rds}}

\hypertarget{results}{%
\section{Results}\label{results}}

\hypertarget{discussion}{%
\section{Discussion}\label{discussion}}

\hypertarget{limitations}{%
\subsection{Limitations}\label{limitations}}

\hypertarget{conclusions}{%
\subsection{Conclusions}\label{conclusions}}

\hypertarget{acknowledgements}{%
\section{Acknowledgements}\label{acknowledgements}}

BMM was funded by the Natural Environment Research Council (NERC) via the IAPETUS2 Doctoral Training Partnership.

\hypertarget{software-availablity}{%
\section{Software availablity}\label{software-availablity}}

In addition to packages already mentioned in the methods we also used the following.

We used \emph{R} v.4.2.2 (\protect\hyperlink{ref-base}{\textbf{base?}}) via \emph{RStudio} v.2023.6.2.561 (\protect\hyperlink{ref-rstudio}{RStudio Team, 2022}).
We used \emph{here} v.1.0.1 (\protect\hyperlink{ref-here}{Müller, 2020}) and \emph{qs} v.0.25.5 (\protect\hyperlink{ref-qs}{Ching, 2023}) to manage directory addresses and saved objects.

We used \emph{raster} v.3.6.14 (\protect\hyperlink{ref-raster}{Hijmans, 2023}) and \emph{RandomFields} v.3.3.14 (\protect\hyperlink{ref-RandomFields}{\textbf{RandomFields?}}) to aid landscape raster creation alongside NLMR v.1.1.1 (\protect\hyperlink{ref-NLMR}{Sciaini et al., 2018}).

We used \emph{ggplot2} v.3.4.2 for creating figures (\protect\hyperlink{ref-ggplot2}{Wickham, 2016}), with the expansions: \emph{patchwork} v.1.1.2 (\protect\hyperlink{ref-patchwork}{\textbf{patchwork?}}), \emph{ggridges} v.0.5.4 (\protect\hyperlink{ref-ggridges}{\textbf{ggridges?}}), and \emph{ggdist} v.3.2.0 (\protect\hyperlink{ref-ggdist}{Kay, 2023a}).

We used \emph{brms} v.2.19.0 (\protect\hyperlink{ref-brms}{\textbf{brms?}}) to run Bayesian models, with diagnostics generated used \emph{bayesplot} v.1.10.0 (\protect\hyperlink{ref-bayesplot}{\textbf{bayesplot?}}), \emph{tidybayes} v.3.0.2 (\protect\hyperlink{ref-tidybayes}{Kay, 2023b}), and \emph{performance} v.0.10.2 (\protect\hyperlink{ref-performance}{\textbf{performance?}}).

We used the \emph{dplyr} v.1.1.3 (\protect\hyperlink{ref-dplyr}{Wickham et al., 2023}), \emph{tibble} v.3.2.1 (\protect\hyperlink{ref-tibble}{Müller \& Wickham, 2023}),
and \emph{stringr} v.1.5.0 (\protect\hyperlink{ref-stringr}{Wickham, 2022}) packages for data manipulation.

We used \emph{sp} v.1.5.1 (\protect\hyperlink{ref-sp}{\textbf{sp?}}), \emph{move} v.4.1.12 (\protect\hyperlink{ref-move}{Kranstauber, Smolla \& Scharf, 2023}) for manipulation of spatial data and estimation of space use not otherwise mentioned in the methods.

We used rmarkdown v.2.25 (\protect\hyperlink{ref-rmarkdown2018}{Xie, Allaire \& Grolemund, 2018}; \protect\hyperlink{ref-rmarkdown2020}{Xie, Dervieux \& Riederer, 2020}; \protect\hyperlink{ref-rmarkdown2023}{Allaire et al., 2023}), bookdown v.0.33 (\protect\hyperlink{ref-bookdown2016}{Xie, 2016}, \protect\hyperlink{ref-R-bookdown}{2022}), tinytex v.0.44 (\protect\hyperlink{ref-tinytex2019}{Xie, 2019}, \protect\hyperlink{ref-tinytex2023}{2023a}), and knitr v.1.41 (\protect\hyperlink{ref-knitr2014}{Xie, 2014}, \protect\hyperlink{ref-knitr2015}{2015}, \protect\hyperlink{ref-knitr2023}{2023b}) packages to generate type-set outputs.

We generated R package citations with the aid of \emph{grateful} v.0.1.13 (\protect\hyperlink{ref-grateful}{Francisco Rodríguez-Sánchez, Connor P. Jackson \& Shaurita D. Hutchins, 2023}).

\hypertarget{data-availabilty}{%
\section{Data availabilty}\label{data-availabilty}}

\hypertarget{supplementary-material}{%
\section{Supplementary Material}\label{supplementary-material}}

\hypertarget{references}{%
\section*{References}\label{references}}
\addcontentsline{toc}{section}{References}

\hypertarget{refs}{}
\begin{CSLReferences}{1}{0}
\leavevmode\vadjust pre{\hypertarget{ref-alberts_self-correction_2015}{}}%
Alberts B, Cicerone RJ, Fienberg SE, Kamb A, McNutt M, Nerem RM, Schekman R, Shiffrin R, Stodden V, Suresh S, Zuber MT, Pope BK, Jamieson KH. 2015. Self-correction in science at work. \emph{Science} 348:1420--1422. DOI: \href{https://doi.org/10.1126/science.aab3847}{10.1126/science.aab3847}.

\leavevmode\vadjust pre{\hypertarget{ref-rmarkdown2023}{}}%
Allaire J, Xie Y, Dervieux C, McPherson J, Luraschi J, Ushey K, Atkins A, Wickham H, Cheng J, Chang W, Iannone R. 2023. \emph{\href{https://github.com/rstudio/rmarkdown}{{rmarkdown}: Dynamic documents for r}}.

\leavevmode\vadjust pre{\hypertarget{ref-altobelli_methods_2022}{}}%
Altobelli JT, Dickinson KJM, Godfrey SS, Bishop PJ. 2022. Methods in amphibian biotelemetry: {Two} decades in review. \emph{Austral Ecology} 47:1382--1395. DOI: \href{https://doi.org/10.1111/aec.13227}{10.1111/aec.13227}.

\leavevmode\vadjust pre{\hypertarget{ref-barto_dissemination_2012}{}}%
Barto EK, Rillig MC. 2012. Dissemination biases in ecology: Effect sizes matter more than quality. \emph{Oikos} 121:228--235. DOI: \href{https://doi.org/10.1111/j.1600-0706.2011.19401.x}{10.1111/j.1600-0706.2011.19401.x}.

\leavevmode\vadjust pre{\hypertarget{ref-qs}{}}%
Ching T. 2023. \emph{\href{https://CRAN.R-project.org/package=qs}{{qs}: Quick serialization of r objects}}.

\leavevmode\vadjust pre{\hypertarget{ref-desbureaux_subjective_2021}{}}%
Desbureaux S. 2021. Subjective modeling choices and the robustness of impact evaluations in conservation science. \emph{Conservation Biology} 35:1615--1626. DOI: \href{https://doi.org/10.1111/cobi.13728}{10.1111/cobi.13728}.

\leavevmode\vadjust pre{\hypertarget{ref-grateful}{}}%
Francisco Rodríguez-Sánchez, Connor P. Jackson, Shaurita D. Hutchins. 2023. \emph{\href{https://github.com/Pakillo/grateful}{{grateful}: Facilitate citation of r packages}}.

\leavevmode\vadjust pre{\hypertarget{ref-gould_same_2023}{}}%
Gould E, Fraser H, Parker T, Nakagawa S, Griffith S, Vesk P, Fidler F, Abbey-Lee R, Abbott J, Aguirre L, Alcaraz C, Altschul D, Arekar K, Atkins J, Atkinson J, Barrett M, Bell K, Bello S, Berauer B, Bertram M, Billman P, Blake C, Blake S, Bliard L, Bonisoli-Alquati A, Bonnet T, Bordes C, Bose A, Botterill-James T, Boyd M, Boyle S, Bradfer-Lawrence T, Brand J, Brengdahl M, Bulla M, Bussière L, Camerlenghi E, Campbell S, Campos L, Caravaggi A, Cardoso P, Carroll C, Catanach T, Chen X, Chik HYJ, Choy E, Christie A, Chuang A, Chunco A, Clark B, Cox M, Cressman K, Crouch C, D'Amelio P, De Sousa A, Döbert T, Dobler R, Dobson A, Doherty T, Drobniak S, Duffy A, Dunn R, Dunning J, Eberhart-Hertel L, Elmore J, Elsherif M, English H, Ensminger D, Ernst U, Ferguson S, Ferreira-Arruda T, Fieberg J, Finch E, Fiorenza E, Fisher D, Forstmeier W, Fourcade Y, Francesca Santostefano F, Frank G, Freund C, Gandy S, Gannon D, García-Cervigón A, Géron C, Gilles M, Girndt A, Gliksman D, Goldspiel H, Gomes D, Goslee S, Gosnell J, Gratton P, Grebe N, Greenler S, Griffith D, Griffith F, Grossman J, Güncan A, Haesen S, Hagan J, Harrison N, Hasnain S, Havird J, Heaton A, Hsu B-Y, Iranzo E, Iverson E, Jimoh S, Johnson D, Johnsson M, Jorna J, Jucker T, Jung M, Kačergytė I, Ke A, Kelly C, Keogan K, Keppeler F, Killion A, Kim D, Kochan D, Korsten P, Kothari S, Kuppler J, Kusch J, Lagisz M, Larkin D, Larson C, Lauck K, Lauterbur M, Law A, Léandri-Breton D-J, Lievens E, Lima D, Lindsay S, Macphie K, Mair M, Malm L, Mammola S, Manhart M, Mäntylä E, Marchand P, Marshall B, Martin D, Martin J, Martin C, Martinig A, McCallum E, McNew S, Meiners S, Michelangeli M, Moiron M, Moreira B, Mortensen J, Mos B, Muraina T, Nelli L, Nilsonne G, Nolazco S, Nooten S, Novotny J, Olin A, Organ C, Ostevik K, Palacio F, Paquet M, Pascall D, Pasquarella V, Payo-Payo A, Pedersen K, Perez G, Perry K, Pottier P, Proulx M, Proulx R, Pruett J, Ramananjato V, Randimbiarison F, Razafindratsima O, Rennison D, Riva F, Riyahi S, Roast M, Rocha F, Roche D, Román-Palacios C, Rosenberg M, Ross J, Rowland F, Rugemalila D, Russell A, Ruuskanen S, Saccone P, Sadeh A, Salazar S, Sales K, Salmón P, Sanchez-Tojar A, Santos L, Schilling H, Schmidt M, Schmoll T, Schneider A, Schrock A, Schroeder J, Schtickzelle N, Schultz N, Scott D, Shapiro J, Sharma N, Shearer C, Sitvarin M, Skupien F, Slinn H, Smith J, Smith G, Sollmann R, Stack Whitney K, Still S, Stuber E, Sutton G, Swallow B, Taff C, Takola E, Tanentzap A, Thawley C, Tortorelli C, Trlica A, Turnell B, Urban L, Van De Vondel S, Van Oordt F, Vanderwel M, Vanderwel K, Vanderwolf K, Verrelli B, Vieira M, Vollering J, Walker X, Walter J, Waryszak P, Weaver R, Weller D, Whelan S, White R, Wolfson D, Wood A, Yanco S, Yen J, Youngflesh C, Zilio G, Zimmer C, Zitomer R, Villamil N, Tompkins E. 2023. Same data, different analysts: Variation in effect sizes due to analytical decisions in ecology and evolutionary biology. \emph{EcoEvoRxiv}. DOI: \href{https://doi.org/10.32942/X2GG62}{10.32942/X2GG62}.

\leavevmode\vadjust pre{\hypertarget{ref-raster}{}}%
Hijmans RJ. 2023. \emph{\href{https://CRAN.R-project.org/package=raster}{{raster}: Geographic data analysis and modeling}}.

\leavevmode\vadjust pre{\hypertarget{ref-hodges_malayan_2022}{}}%
Hodges CW, Marshall BM, Hill JG, Strine CT. 2022. Malayan kraits ({Bungarus} candidus) show affinity to anthropogenic structures in a human dominated landscape. \emph{Scientific Reports} 12:7139. DOI: \href{https://doi.org/10.1038/s41598-022-11255-z}{10.1038/s41598-022-11255-z}.

\leavevmode\vadjust pre{\hypertarget{ref-huntingtonklein_influence_2021}{}}%
Huntington‐Klein N, Arenas A, Beam E, Bertoni M, Bloem JR, Burli P, Chen N, Grieco P, Ekpe G, Pugatch T, Saavedra M, Stopnitzky Y. 2021. The influence of hidden researcher decisions in applied microeconomics. \emph{Economic Inquiry} 59:944--960. DOI: \href{https://doi.org/10.1111/ecin.12992}{10.1111/ecin.12992}.

\leavevmode\vadjust pre{\hypertarget{ref-jennions_relationships_2002}{}}%
Jennions MD, Møller AP. 2002. Relationships fade with time: A meta-analysis of temporal trends in publication in ecology and evolution. \emph{Proceedings of the Royal Society of London. Series B: Biological Sciences} 269:43--48. DOI: \href{https://doi.org/10.1098/rspb.2001.1832}{10.1098/rspb.2001.1832}.

\leavevmode\vadjust pre{\hypertarget{ref-joo_recent_2022}{}}%
Joo R, Picardi S, Boone ME, Clay TA, Patrick SC, Romero-Romero VS, Basille M. 2022. Recent trends in movement ecology of animals and human mobility. \emph{Movement Ecology} 10:26. DOI: \href{https://doi.org/10.1186/s40462-022-00322-9}{10.1186/s40462-022-00322-9}.

\leavevmode\vadjust pre{\hypertarget{ref-ggdist}{}}%
Kay M. 2023a. \emph{{ggdist}: Visualizations of distributions and uncertainty}. DOI: \href{https://doi.org/10.5281/zenodo.3879620}{10.5281/zenodo.3879620}.

\leavevmode\vadjust pre{\hypertarget{ref-tidybayes}{}}%
Kay M. 2023b. \emph{{tidybayes}: Tidy data and geoms for {Bayesian} models}. DOI: \href{https://doi.org/10.5281/zenodo.1308151}{10.5281/zenodo.1308151}.

\leavevmode\vadjust pre{\hypertarget{ref-kays_movebank_2022}{}}%
Kays R, Davidson SC, Berger M, Bohrer G, Fiedler W, Flack A, Hirt J, Hahn C, Gauggel D, Russell B, Kölzsch A, Lohr A, Partecke J, Quetting M, Safi K, Scharf A, Schneider G, Lang I, Schaeuffelhut F, Landwehr M, Storhas M, Schalkwyk L, Vinciguerra C, Weinzierl R, Wikelski M. 2022. The {Movebank} system for studying global animal movement and demography. \emph{Methods in Ecology and Evolution} 13:419--431. DOI: \href{https://doi.org/10.1111/2041-210X.13767}{10.1111/2041-210X.13767}.

\leavevmode\vadjust pre{\hypertarget{ref-kelly_rate_2019}{}}%
Kelly CD. 2019. Rate and success of study replication in ecology and evolution. \emph{PeerJ} 7:e7654. DOI: \href{https://doi.org/10.7717/peerj.7654}{10.7717/peerj.7654}.

\leavevmode\vadjust pre{\hypertarget{ref-knierim_spatial_2019}{}}%
Knierim T. 2019. Spatial ecology study reveals nest attendance and habitat preference of banded kraits ({Bungarus} fasciatus). \emph{Herpetological Bulletin}:6--13. DOI: \href{https://doi.org/10.33256/hb150.613}{10.33256/hb150.613}.

\leavevmode\vadjust pre{\hypertarget{ref-move}{}}%
Kranstauber B, Smolla M, Scharf AK. 2023. \emph{\href{https://CRAN.R-project.org/package=move}{{move}: Visualizing and analyzing animal track data}}.

\leavevmode\vadjust pre{\hypertarget{ref-marshall_no_2020}{}}%
Marshall BM, Crane M, Silva I, Strine CT, Jones MD, Hodges CW, Suwanwaree P, Artchawakom T, Waengsothorn S, Goode M. 2020. No room to roam: {King} {Cobras} reduce movement in agriculture. \emph{Movement Ecology} 8:33. DOI: \href{https://doi.org/10.1186/s40462-020-00219-5}{10.1186/s40462-020-00219-5}.

\leavevmode\vadjust pre{\hypertarget{ref-Marshall2018}{}}%
Marshall BM, Strine CT, Jones MD, Artchawakom T, Silva I, Suwanwaree P, Goode M. 2019. Space fit for a king: Spatial ecology of king cobras ({Ophiophagus} hannah) in {Sakaerat} {Biosphere} {Reserve}, {Northeastern} {Thailand}. \emph{Amphibia-Reptilia} 40:163--178. DOI: \href{https://doi.org/10.1163/15685381-18000008}{10.1163/15685381-18000008}.

\leavevmode\vadjust pre{\hypertarget{ref-here}{}}%
Müller K. 2020. \emph{\href{https://CRAN.R-project.org/package=here}{{here}: A simpler way to find your files}}.

\leavevmode\vadjust pre{\hypertarget{ref-tibble}{}}%
Müller K, Wickham H. 2023. \emph{\href{https://CRAN.R-project.org/package=tibble}{{tibble}: Simple data frames}}.

\leavevmode\vadjust pre{\hypertarget{ref-nakagawa_replicating_2015}{}}%
Nakagawa S, Parker TH. 2015. Replicating research in ecology and evolution: Feasibility, incentives, and the cost-benefit conundrum. \emph{BMC Biology} 13:88. DOI: \href{https://doi.org/10.1186/s12915-015-0196-3}{10.1186/s12915-015-0196-3}.

\leavevmode\vadjust pre{\hypertarget{ref-noonan_effects_2020}{}}%
Noonan MJ, Fleming CH, Tucker MA, Kays R, Harrison A, Crofoot MC, Abrahms B, Alberts SC, Ali AH, Altmann J, Antunes PC, Attias N, Belant JL, Beyer DE, Bidner LR, Blaum N, Boone RB, Caillaud D, Paula RC, Torre JA la, Dekker J, DePerno CS, Farhadinia M, Fennessy J, Fichtel C, Fischer C, Ford A, Goheen JR, Havmøller RW, Hirsch BT, Hurtado C, Isbell LA, Janssen R, Jeltsch F, Kaczensky P, Kaneko Y, Kappeler P, Katna A, Kauffman M, Koch F, Kulkarni A, LaPoint S, Leimgruber P, Macdonald DW, Markham AC, McMahon L, Mertes K, Moorman CE, Morato RG, Moßbrucker AM, Mourão G, O'Connor D, Oliveira‐Santos LGR, Pastorini J, Patterson BD, Rachlow J, Ranglack DH, Reid N, Scantlebury DM, Scott DM, Selva N, Sergiel A, Songer M, Songsasen N, Stabach JA, Stacy‐Dawes J, Swingen MB, Thompson JJ, Ullmann W, Vanak AT, Thaker M, Wilson JW, Yamazaki K, Yarnell RW, Zieba F, Zwijacz‐Kozica T, Fagan WF, Mueller T, Calabrese JM. 2020. Effects of body size on estimation of mammalian area requirements. \emph{Conservation Biology}:cobi.13495. DOI: \href{https://doi.org/10.1111/cobi.13495}{10.1111/cobi.13495}.

\leavevmode\vadjust pre{\hypertarget{ref-peterson_self-correction_2021}{}}%
Peterson D, Panofsky A. 2021. Self-correction in science: {The} diagnostic and integrative motives for replication. \emph{Social Studies of Science}.

\leavevmode\vadjust pre{\hypertarget{ref-portugal_externally_2022}{}}%
Portugal SJ, White CR. 2022. Externally attached biologgers cause compensatory body mass loss in birds. \emph{Methods in Ecology and Evolution} 13:294--302. DOI: \href{https://doi.org/10.1111/2041-210X.13754}{10.1111/2041-210X.13754}.

\leavevmode\vadjust pre{\hypertarget{ref-robstad_impact_2021}{}}%
Robstad CA, Lodberg-Holm HK, Mayer M, Rosell F. 2021. The impact of bio-logging on body weight change of the {Eurasian} beaver. \emph{PLOS ONE} 16:e0261453. DOI: \href{https://doi.org/10.1371/journal.pone.0261453}{10.1371/journal.pone.0261453}.

\leavevmode\vadjust pre{\hypertarget{ref-rstudio}{}}%
RStudio Team. 2022. \emph{\href{http://www.rstudio.com/}{{RStudio}: Integrated development environment for r}}. Boston, MA: RStudio, PBC.

\leavevmode\vadjust pre{\hypertarget{ref-salis_how_2021}{}}%
Salis A, Lena J-P, Lengagne T. 2021. How {Subtle} {Protocol} {Choices} {Can} {Affect} {Biological} {Conclusions}: {Great} {Tits}' {Response} to {Allopatric} {Mobbing} {Calls}. \emph{Animal Behavior and Cognition} 8:152--165. DOI: \href{https://doi.org/10.26451/abc.08.02.05.2021}{10.26451/abc.08.02.05.2021}.

\leavevmode\vadjust pre{\hypertarget{ref-NLMR}{}}%
Sciaini M, Fritsch M, Scherer C, Simpkins CE. 2018. \href{https://doi.org/10.1111/2041-210X.13076}{NLMR and landscapetools: An integrated environment for simulating and modifying neutral landscape models in r}. \emph{Methods in Ecololgy and Evolution} 00:1--9.

\leavevmode\vadjust pre{\hypertarget{ref-silberzahn_many_2018}{}}%
Silberzahn R, Uhlmann EL, Martin DP, Anselmi P, Aust F, Awtrey E, Bahník Š, Bai F, Bannard C, Bonnier E, Carlsson R, Cheung F, Christensen G, Clay R, Craig MA, Dalla Rosa A, Dam L, Evans MH, Flores Cervantes I, Fong N, Gamez-Djokic M, Glenz A, Gordon-McKeon S, Heaton TJ, Hederos K, Heene M, Hofelich Mohr AJ, Högden F, Hui K, Johannesson M, Kalodimos J, Kaszubowski E, Kennedy DM, Lei R, Lindsay TA, Liverani S, Madan CR, Molden D, Molleman E, Morey RD, Mulder LB, Nijstad BR, Pope NG, Pope B, Prenoveau JM, Rink F, Robusto E, Roderique H, Sandberg A, Schlüter E, Schönbrodt FD, Sherman MF, Sommer SA, Sotak K, Spain S, Spörlein C, Stafford T, Stefanutti L, Tauber S, Ullrich J, Vianello M, Wagenmakers E-J, Witkowiak M, Yoon S, Nosek BA. 2018. Many {Analysts}, {One} {Data} {Set}: {Making} {Transparent} {How} {Variations} in {Analytic} {Choices} {Affect} {Results}. \emph{Advances in Methods and Practices in Psychological Science} 1:337--356. DOI: \href{https://doi.org/10.1177/2515245917747646}{10.1177/2515245917747646}.

\leavevmode\vadjust pre{\hypertarget{ref-smith_native_2021}{}}%
Smith SN, Jones MD, Marshall BM, Waengsothorn S, Gale GA, Strine CT. 2021. Native {Burmese} pythons exhibit site fidelity and preference for aquatic habitats in an agricultural mosaic. \emph{Scientific Reports} 11:7014. DOI: \href{https://doi.org/10.1038/s41598-021-86640-1}{10.1038/s41598-021-86640-1}.

\leavevmode\vadjust pre{\hypertarget{ref-stuber_spatial_2022}{}}%
Stuber EF, Carlson BS, Jesmer BR. 2022. Spatial personalities: A meta-analysis of consistent individual differences in spatial behavior. \emph{Behavioral Ecology} 33:477--486. DOI: \href{https://doi.org/10.1093/beheco/arab147}{10.1093/beheco/arab147}.

\leavevmode\vadjust pre{\hypertarget{ref-tomotani_great_2021}{}}%
Tomotani BM, Muijres FT, Johnston B, Jeugd HP, Naguib M. 2021. Great tits do not compensate over time for a radio‐tag‐induced reduction in escape‐flight performance. \emph{Ecology and Evolution} 11:16600--16617. DOI: \href{https://doi.org/10.1002/ece3.8240}{10.1002/ece3.8240}.

\leavevmode\vadjust pre{\hypertarget{ref-Weatherhead2004}{}}%
Weatherhead PJ, Blouin-Demers G. 2004. Long-term effects of radiotelemetry on black ratsnakes. \emph{Wildlife Society Bulletin} 32:900--906. DOI: \href{https://doi.org/10.2193/0091-7648(2004)032\%5B0900:LEOROB\%5D2.0.CO;2}{10.2193/0091-7648(2004)032{[}0900:LEOROB{]}2.0.CO;2}.

\leavevmode\vadjust pre{\hypertarget{ref-ggplot2}{}}%
Wickham H. 2016. \emph{\href{https://ggplot2.tidyverse.org}{ggplot2: Elegant graphics for data analysis}}. Springer-Verlag New York.

\leavevmode\vadjust pre{\hypertarget{ref-stringr}{}}%
Wickham H. 2022. \emph{\href{https://CRAN.R-project.org/package=stringr}{{stringr}: Simple, consistent wrappers for common string operations}}.

\leavevmode\vadjust pre{\hypertarget{ref-dplyr}{}}%
Wickham H, François R, Henry L, Müller K, Vaughan D. 2023. \emph{\href{https://CRAN.R-project.org/package=dplyr}{{dplyr}: A grammar of data manipulation}}.

\leavevmode\vadjust pre{\hypertarget{ref-knitr2014}{}}%
Xie Y. 2014. {knitr}: A comprehensive tool for reproducible research in {R}. In: Stodden V, Leisch F, Peng RD eds. \emph{Implementing reproducible computational research}. Chapman; Hall/CRC,.

\leavevmode\vadjust pre{\hypertarget{ref-knitr2015}{}}%
Xie Y. 2015. \emph{\href{https://yihui.org/knitr/}{Dynamic documents with {R} and knitr}}. Boca Raton, Florida: Chapman; Hall/CRC.

\leavevmode\vadjust pre{\hypertarget{ref-bookdown2016}{}}%
Xie Y. 2016. \emph{\href{https://bookdown.org/yihui/bookdown}{{bookdown}: Authoring books and technical documents with {R} markdown}}. Boca Raton, Florida: Chapman; Hall/CRC.

\leavevmode\vadjust pre{\hypertarget{ref-tinytex2019}{}}%
Xie Y. 2019. \href{https://tug.org/TUGboat/Contents/contents40-1.html}{{TinyTeX}: A lightweight, cross-platform, and easy-to-maintain LaTeX distribution based on TeX live}. \emph{TUGboat} 40:30--32.

\leavevmode\vadjust pre{\hypertarget{ref-R-bookdown}{}}%
Xie Y. 2022. \emph{\href{https://CRAN.R-project.org/package=bookdown}{Bookdown: Authoring books and technical documents with r markdown}}.

\leavevmode\vadjust pre{\hypertarget{ref-knitr2023}{}}%
Xie Y. 2023b. \emph{\href{https://yihui.org/knitr/}{{knitr}: A general-purpose package for dynamic report generation in r}}.

\leavevmode\vadjust pre{\hypertarget{ref-tinytex2023}{}}%
Xie Y. 2023a. \emph{\href{https://github.com/rstudio/tinytex}{{tinytex}: Helper functions to install and maintain TeX live, and compile LaTeX documents}}.

\leavevmode\vadjust pre{\hypertarget{ref-rmarkdown2018}{}}%
Xie Y, Allaire JJ, Grolemund G. 2018. \emph{\href{https://bookdown.org/yihui/rmarkdown}{R markdown: The definitive guide}}. Boca Raton, Florida: Chapman; Hall/CRC.

\leavevmode\vadjust pre{\hypertarget{ref-rmarkdown2020}{}}%
Xie Y, Dervieux C, Riederer E. 2020. \emph{\href{https://bookdown.org/yihui/rmarkdown-cookbook}{R markdown cookbook}}. Boca Raton, Florida: Chapman; Hall/CRC.

\end{CSLReferences}

\end{document}
